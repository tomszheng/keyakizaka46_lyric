%----------------------------------------------------------------------------
% 设置页面的大小
\setlength\paperheight{210mm}
\setlength\paperwidth{148mm}

%----------------------------------------------------------------------------
% 设置页面的边距
\usepackage{geometry}
\geometry{left=1cm,right=1cm,top=2cm,bottom=2cm}

%----------------------------------------------------------------------------
% 用于使目录能被点击
\usepackage[bookmarks=true,colorlinks,linkcolor=black]{hyperref}
\renewcommand{\contentsname}{\centerline{mulu}}

%----------------------------------------------------------------------------
% 将 `第X章` -> `x`
\CTEXsetup[name={,}, number=\arabic{chapter}]{chapter}

%----------------------------------------------------------------------------
% 双线页眉的设置
\makeatletter %双线页眉
\def\headrule{{\if@fancyplain\let\headrulewidth\plainheadrulewidth\fi%
\hrule\@height 0.5pt\@width\headwidth  %下面0.5pt粗
\vskip-2\headrulewidth\vskip-1pt}      %两条线的距离1pt
\vspace{6mm}}                          %双线与下面正文之间的垂直间距
\makeatother

%----------------------------------------------------------------------------
% 对于中文的支持
\usepackage{xeCJK}
\renewcommand\CJKfamilydefault{\CJKsfdefault}
\setCJKsansfont{SimSun} % 宋体
%\setCJKsansfont{Microsoft YaHei} % 微软雅黑

%----------------------------------------------------------------------------
% 对于日文的支持
\usepackage{zxjatype}
\setjamainfont{ipaexm.ttf}

%----------------------------------------------------------------------------
% 对于注音的支持
\usepackage[overlap,CJK]{ruby}

%----------------------------------------------------------------------------
% 用于行距修改
\usepackage{setspace}

%----------------------------------------------------------------------------
% 晕倒,这是做什么用的,貌似不能少啊...  TomsZheng 2016-1-3
\usepackage[english]{babel}

%----------------------------------------------------------------------------
% 使用多种颜色
% 关于颜色可以上该网站上进行查询
%     http://latexcolor.com/
\usepackage[usenames,dvipsnames]{color}

%----------------------------------------------------------------------------
% 对花体的支持
\usepackage{mathrsfs}

%----------------------------------------------------------------------------
% 修改章节号的样式
% 用到了 `使用多种颜色` & `对花体的支持`

\usepackage{titletoc}
\usepackage[sf,sl,outermarks]{titlesec}

%----------------------------------------------------------------------------
% 用于在页面上绘图
\usepackage{tikz}
\usetikzlibrary{calc} %用于绘图时的相应坐标计算,这里我还没弄明白... 2016-1-11 TomsZheng

%----------------------------------------------------------------------------
% 用于 if-else表达、判断奇偶页
\usepackage{ifthen,changepage}

%----------------------------------------------------------------------------
% 修改 Chapter 的样式,主要的生成效果请看生成的文档
\titleformat{\chapter}[display]
    {}
    {\hfill \tikz[remember picture] \node[] (nr) {};
    \begin{tikzpicture}[overlay,remember picture]
        \coordinate (rightborder) at ($(nr)+(98,0)$);
        \coordinate (right) at ($(nr.east) + (0,0)$);
        \color{BrickRed} \draw[line width=3.5em] (right) -- (rightborder); % 这里的绘制方式还没弄明白 2016-1-11 TomsZheng
    \end{tikzpicture}}
    {-1ex}
    {\filcenter\fontsize{30}{50}\selectfont}
    [\vspace{-1ex}]

% 修改 Section 的样式,主要的生成效果请看生成的文档
\titleformat{\section}[display]
    {}
    {\hfill \tikz[remember picture]  \node[] (nr) at (0,0)  {};
    \begin{tikzpicture}[remember picture,overlay]
        \checkoddpage % 判断当前是处于奇数页还是偶数页,不同的情况下采用不同的 Style [针对 Section 而言]
        \ifthenelse{\boolean{oddpage}}
        {
            \coordinate (rightborder) at ($(nr)+(100,0)$);
            \coordinate (right) at ($(nr.west) + (0,0)$);
            \color{Purple} \draw[line width=2.3em] (right) -- (rightborder); % 这里的绘制方式还没弄明白 2016-1-11 TomsZheng
        }
        {
            \coordinate (leftborder) at ($(nr)-(\paperwidth-78,0)$);
            \coordinate (left) at ($(nr.east)-(\paperwidth,0)$);
            \color{ForestGreen} \draw[line width=2.3em] (left) -- (leftborder); % 这里的绘制方式还没弄明白 2016-1-11 TomsZheng
        }
    \end{tikzpicture}}
    {-3.5ex}
    {\filcenter\LARGE\selectfont}
    [\vspace{0ex}]

%----------------------------------------------------------------------------
% 用于修改缩进模式
\usepackage{indentfirst}
\setlength{\parindent}{2em} % 首行缩进

%----------------------------------------------------------------------------
% 用于页眉与页脚的定制
\usepackage{fancyhdr}

\pagestyle{fancy} % 这一句一定不能少,否则下面的两句将不会生效
\renewcommand{\chaptermark}[1]{\markboth{#1}{}}
\renewcommand{\sectionmark}[1]{\markright{#1}{}}

\fancyfoot{}
\fancyfoot[EL,OR]{\thepage} % 奇偶页的页码在底部的不同位置

\fancyhead{}
\fancyhead[OR]{\nouppercase{\leftmark}} % 奇数页的 head 的右侧位置为歌手名
\fancyhead[EL]{\nouppercase{\leftmark}} % 偶数页的 head 的左侧位置为歌手名
\fancyhead[EC,OC]{\rightmark} % 奇偶页的 head 的居中位置为歌曲名

%----------------------------------------------------------------------------
% 用于对花体字母的支持
\usepackage{mathrsfs}

%----------------------------------------------------------------------------
% 用于控制脚标的缩进
\usepackage[hang,flushmargin]{footmisc}

%----------------------------------------------------------------------------
% 用于修改脚注的上标的颜色
\renewcommand{\thefootnote}{\textcolor{ForestGreen}{\arabic{footnote}}}

\usepackage{graphicx}
\usepackage{caption}
\usepackage{subcaption}

%----------------------------------------------------------------------------
% 用于双语的对照

\usepackage{parallel}	% bilingual text in two columns

% Macros
% ---------------
\newcommand{\SecondLanguageSection}[1]{
\begin{center}
  \large{\textnormal{#1}}\normalfont\normalsize
\end{center}
}	% for the translated section title
